% !TeX root=tukedip.tex
% !TeX encoding = UTF-8
% !TeX spellcheck = sk_SK
\section{Zhrnutie}
\textbf{1. Vytvoril funkčný React project, ktorý vykresľuje používateľské rozhranie pomocou niekoľkých háčikov React, ako sú useState, useEffect a useContext.}

Použitie funkčných komponentov a háčikov React umožňuje efektívnejší a flexibilnejší kód v porovnaní s komponentmi založenými na triedach.
Háčik useState sa používa na správu stavových údajov v rámci komponentu.
Háčik useEffect sa používa na spracovanie vedľajších efektov, ako je napríklad aktualizácia používateľského rozhrania v reakcii na zmeny údajov alebo vykonávanie požiadaviek API.
useContext hook sa používa na zdieľanie údajov medzi komponentmi bez toho, aby bolo potrebné odovzdávať rekvizity nadol cez viaceré úrovne.

\textbf{2. Na vylepšenie používateľského rozhrania sa využívajú rôzne komponenty UI Material, ako napríklad Grid, ToggleButton, Tabs, Tab, Button, Typography a Box.}

Material UI poskytuje knižnicu vopred pripravených komponentov, ktoré možno ľahko prispôsobiť a naštýlovať tak, aby zodpovedali požadovanému dizajnu používateľského rozhrania.
Komponent Grid sa používa na vytvorenie flexibilného a citlivého rozvrhnutia používateľského rozhrania.
Komponenty ToggleButton, Tabs a Tab sa používajú na vytvorenie navigačného systému používateľského rozhrania založeného na kartách.
Komponenty Button a Typography sa používajú na vytvorenie rôznych tlačidiel a textových prvkov pre používateľské rozhranie.
Komponent Box sa používa na vytvorenie kontajnerového prvku, ktorý možno ľahko štylizovať a umiestňovať v rámci používateľského rozhrania.

\textbf{3. Implementovaná vlastná komponenta BatteryGauge a komponenta ControlBlock, ktorá vykresľuje niekoľko komponentov NavigationButton, čo pridáva UI ďalšie funkcie.}

Komponent BatteryGauge sa používa na zobrazenie úrovne nabitia batérie dronu v grafickej podobe.
Komponent ControlBlock sa používa na zoskupenie niekoľkých komponentov NavigationButton, aby bolo používateľské rozhranie kompaktnejšie a usporiadanejšie.
Komponent NavigationButton sa používa na vytvorenie rôznych tlačidiel na ovládanie dronu, napríklad na vzlet, pristátie a núdzové zastavenie.

\textbf{4. Úspešné pripojenie k serveru WebSocket pomocou knižníc useContext a socket.io-client, čo umožňuje programu prijímať a odosielať údaje na server.}

WebSocket je protokol, ktorý umožňuje komunikáciu medzi klientom a serverom v reálnom čase.
Háčik useContext sa používa na zdieľanie objektu pripojenia WebSocket medzi viacerými komponentmi.
Knižnica socket.io-client sa používa na spracovanie pripojenia a komunikácie WebSocket.
Počúval na udalosti pomocou useEffect a aktualizoval stav prijatými údajmi, čím umožnil aktualizáciu používateľského rozhrania v reálnom čase.

\textbf{5. Implementovaná detekcia značiek Aruco a výpočet polohy pomocou knižnice OpenCV, čo umožňuje programu presne určiť polohu dronu vzhľadom na značky.}

Vyvinutý systém je na základe výsledkov testov schopný riadiť dron v správne navrhnutom prostredí. Merania polohy dronu sú však neisté, v niektorých prípadoch podliehajú veľkým chybám a v iných prípadoch poskytujú pomerne presné hodnoty. Keďže systém funguje na princípe kamery, jeho činnosť ovplyvňujú aj okolité svetelné podmienky. pri písaní programu bolo cieľom, aby bol systém čo najvšeobecnejší a najadaptívnejší. To si samozrejme vyžaduje vytvorenie riadne overeného letového prostredia. Výpočet transformácií medzi súradnicovými systémami značiek vnáša chyby, ktoré majú približne rovnakú šancu vzájomne sa korigovať alebo zväčšovať. Táto časť systému je stále neistá a je potrebné vyvinúť jej vhodnú korekciu. Po stanovení transformácií je možné previesť dve výsledné polohy kamery z markerového do globálneho súradnicového systému. Z týchto výsledkov vyplýva, že odchýlka získaných polôh je v rozsahu ±0,05 m. Tieto odchýlky sa približne zhodujú s chybami merania markerov ArUco, ale tieto odchýlky sa počas prepočtov kumulujú. 

Riešením možných chybných meraní je optimalizácia a zlepšenie rozpoznávania značiek ArUco. To zahŕňa lepšiu detekciu okrajov markerov, ktoré môžu tiež spôsobovať chyby merania, a filtrovanie transformácie PnP na odstránenie osových reverzií. Takéto chyby vedú k nesprávnym bodom merania, ktoré by sa dali odstrániť zlepšením algoritmu. Existujú už vylepšené algoritmy, ktoré sľubujú lepšie filtrovanie v určitých programových prostrediach (napr. ROS). \citep{SRS-aruco} Určenie uhlových polôh a ich prevod na Eulerove uhly si tiež vyžaduje ďalší vývoj a filtrovanie. Žiaľ, 180-stupňová periodicita funkcie atan2() použitej pri transformácii z matice rotácie \citep{Depriester2018} spôsobila, že výsledky sú bez filtrovania nepoužiteľné. Transformácie s nefiltrovanými rotačnými ma-tricami môžu tiež spôsobiť chyby merania, ktoré sa nedajú korigovať použitým priemerovaním.

Použitý dron Tello nie je vzhľadom na svoju veľkosť profesionálne zariadenie, wifi pripojenie je často zašumené a najväčším problémom je oneskorenie obrazu z kamery, ktoré je takmer 2 sekundy. Napriek tomu je odčítanie stavu plynulé. Hoci sa ukázalo, že niektoré snímače niekedy poskytujú nesprávne údaje - napríklad snímanie výšky kamerou ToF - snímače fungujú pre konštrukciu dronu dobre. Z režimov riadenia je použité RC riadenie schopné správne určovať polohu pomocou hodnôt rýchlosti, pričom väčšiu chybu spôsobuje spomínané oneskorenie. program je vhodný na riadenie vnútorného mikropohonu a dokázal úspešne dosiahnuť svoje ciele. Jeho presnosť nie je vždy dostatočná, a to jednak z dôvodu nedostatočného filtrovania programu v reálnom čase, či už z obrazových alebo polohových údajov, a jednak z dôvodu hardvérových nedostatkov. Navrhnutý systém by mohol byť vhodný na navigáciu aj v priemyselnom prostredí väčšieho rozsahu. Napríklad v prípade riadenia dronu v rámci haly, kde sú známe relatívne polohy značiek a dron sa naviguje a zbiera údaje na základe značiek videných v hale.

\subsection{Návrhy na ďalší vývoj}

Ako už bolo uvedené v predchádzajúcom odseku, filtrovanie v reálnom čase je nevyhnutné.  Post-filtrácia meracích bodov nie je dostatočná, ak je chyba v uložení súradnicových systémov značiek. Ak by sa podarilo nastaviť Kalmanov filter v reálnom čase, bolo by dokonca možné zobraziť dátové body bezprostredne po ňom, takže by sa trajektória dronu dala zobraziť v riešení takmer v reálnom čase. prevádzka v reálnom čase si vyžaduje aj softvérový balík CUDA od spoločnosti Nvidia, pomocou ktorého možno načítať matice na grafickú kartu počítača a rýchlejšie vykonávať operácie spracovania matíc a obrazov. Ten v súčasnosti ešte nie je vyvinutý v jazyku Python, ale v blízkej fu-ture môžu byť k dispozícii linkovacie knižnice pre verziu OpenCV v jazyku Python. 

Prípadne by sa celý program mohol prepísať do jazyka C++ alebo C\#, kde už existujú rozhrania CUDA s OpenCV. v práci sa používajú len relatívne lacné drony, a to aj na domáce použitie. Rovnako aj určovanie polohy z ArUco markerov možno považovať za "lacnú" metódu. Samozrejme, s lepším hardvérom sa dá dosiahnuť lepšia práca, ale to si vyžaduje priemyselný alebo domáci dron. Tým by sa odstránilo oneskorenie obrazu z kamery na prop-er vyhýbanie sa prekážkam. S rýchlejším bezdrôtovým pripojením by vysielanie mohlo prichádzať takmer okamžite a ovládanie dronu by mohlo fungovať pri vyšších rýchlostiach. Z rovnakého dôvodu by sa mohla výrazne znížiť chyba nastavenia ovládača. pri väčšej aplikácii by systém mohol byť podporovaný externými údajmi.