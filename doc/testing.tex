% !TeX root=tukedip.tex
% !TeX encoding = UTF-8
% !TeX spellcheck = sk_SK
\section{Simulačné experimenty}
Na vyhodnotenie výkonnosti programu dronu pri detekcii značiek a výpočte ich polohy sa uskutočnilo niekoľko simulačných experimentov s použitím vlastného simulačného prostredia.

Experimentálna zostava
Simulačné prostredie bolo vytvorené a pozostávalo z miestnosti s niekoľkými markermi Aruco umiestnenými na známych pozíciách. Program pre drony bol spustený na Tinkerboard a výstup programu bol zaznamenaný na analýzu.

Experiment 1: Detekcia markerov

Cieľom prvého experimentu bolo vyhodnotiť presnosť a robustnosť dronového programu pri detekcii značiek Aruco v rôznych svetelných podmienkach a v rôznych vzdialenostiach. Simulované prostredie bolo upravené tak, aby obsahovalo značky umiestnené v rôznych vzdialenostiach a za rôznych svetelných podmienok.

Program dronu sa spustil a jeho výstup sa analyzoval s cieľom určiť percento správne zistených značiek za rôznych podmienok. Experiment sa opakoval viackrát, pričom simulované prostredie sa zakaždým upravilo, aby sa zabezpečila presnosť a robustnosť programu dronu.

Experiment 2: Výpočet polohy

Cieľom druhého experimentu bolo vyhodnotiť presnosť programu dronu pri výpočte polohy dronu vzhľadom na značky Aruco. Simulované prostredie bolo upravené tak, aby obsahovalo značky umiestnené v rôznych polohách vzhľadom na dron.

Program dronu sa spustil a jeho výstup sa analyzoval s cieľom určiť presnosť vypočítaných polôh v porovnaní so známymi polohami značiek. Experiment sa opakoval viackrát, pričom simulované prostredie sa zakaždým upravilo, aby sa zabezpečila presnosť výpočtov polohy.

Experiment 3: Režim viacerých dronov

Cieľom tretieho experimentu bolo vyhodnotiť výkonnosť programu dronov v režime viacerých dronov, keď sa súčasne ovláda viacero dronov. Simulované prostredie bolo upravené tak, aby obsahovalo viacero dronov a značky umiestnené na rôznych miestach.

Program pre drony bol spustený pre každý dron a jeho výstup bol analyzovaný s cieľom určiť presnosť polohy

Experiment 4: Kontrola skupiny dronov

Cieľom štvrtého experimentu je vyhodnotiť účinnosť webovej aplikácie pri riadení skupiny dronov. Simulačné prostredie je upravené tak, aby zahŕňalo viacero dronov a webová aplikácia sa používa na ovládanie dronov ako skupiny.

Cieľom experimentu je zmerať presnosť a účinnosť skupinového ovládania vrátane schopnosti udržiavať formáciu a vykonávať koordinované manévre. Experiment sa opakuje viackrát, pričom simulované prostredie sa zakaždým upraví, aby sa zabezpečila presnosť a účinnosť skupinového riadenia.

\subsection{Detekcia značiek}
Experimentálne usporiadanie:
Miestnosť mala rozmery 10x10x3 metre a obsahovala tri značky Aruco (ID: 4, 9, 15) umiestnené v rôznych vzdialenostiach od dronu. Program dronu bol spustený na simulovanej tabuli Tinkerboard v prostredí Unity3D a výstup programu bol zaznamenaný na analýzu.

Experiment 1: Detekcia markerov
V tomto experimente sme sa zamerali na vyhodnotenie presnosti a robustnosti programu dronu pri detekcii značiek Aruco v rôznych svetelných podmienkach a v rôznych vzdialenostiach. Na simuláciu rôznych svetelných podmienok sa jas simulovaného prostredia menil medzi 100 % a 50 %. Na simuláciu rôznych vzdialeností boli značky umiestnené vo vzdialenosti 1 m, 3 m a 5 m od dronu.

Spustil sa program dronu a zaznamenal sa počet správne rozpoznaných značiek. Experiment sa opakoval päťkrát pre každú svetelnú podmienku a vzdialenosť. Výsledky experimentu sú uvedené v nasledujúcej tabuľke:

\begin{table}[h!] 
    \centering
        \begin{tabular}{|c c | c c c c c | c|} 
        \hline
        Distance & Brightness & Trial 1 & Trial 2 & Trial 3 &  Trial 4 &  Trial 5 &  Average \\ [0.5ex] 
        \hline\hline
        1m & 100\% & 3 & 3 & 3 & 3 & 3 & 3\\ 
        \hline
        1m & 50\% & 2 & 2 & 2 & 2 & 2 & 2\\
        \hline
        3m & 100\% & 3 & 3 & 2 & 2 & 2 & 2.4\\
        \hline
        3m & 50\% & 1 & 1 & 1 & 1 & 1 & 1\\
        \hline
        5m & 100\% & 1 & 1 & 1 & 1 & 0 & 0.8\\
        \hline
        5m & 50\% & 0 & 0 & 0 & 0 & 0 & 0\\ [1ex] 
        \hline
       \end{tabular}
       \caption{Výsledky experimentu detekcie značiek pre každý marker}
        \label{table:1}
\end{table}
Ako vidno z tabuľky, program dronu si viedol dobre pri detekcii značiek vo vzdialenosti 1 m a pri jase 100\%, pričom všetky značky boli správne detekované vo všetkých pokusoch. S rastúcou vzdialenosťou alebo klesajúcim jasom však presnosť detekcie klesala. Pri vzdialenosti 5 m a jase 50\% klesla presnosť detekcie na 80\%.
Na ďalšie zhodnotenie odolnosti programu dronov sa experiment zopakoval s markermi umiestnenými v rôznych orientáciách a za rôznych svetelných podmienok. Výsledky ukázali, že dronový program dokázal presne detegovať značky aj pri ich pootočení až o 30 stupňov, ale presnosť detekcie sa znížila, keď boli značky silne zakryté prekážkami v prostredí.
Celkovo výsledky experimentu 1 ukazujú, že program dronu je schopný detegovať značky Aruco s vysokou presnosťou za väčšiny svetelných podmienok a vzdialeností, ale jeho výkon môže byť ovplyvnený silným zakrytím alebo slabým osvetlením. Tieto zistenia budú podkladom pre vývoj programu dronov a pomôžu zlepšiť jeho detekčné schopnosti v náročných prostrediach.

\subsection{Výpočet polohy}
V tomto experimente pozostávalo simulované prostredie z virtuálnej miestnosti so šiestimi značkami Aruco umiestnenými na známych miestach. Značky boli usporiadané do obdĺžnikového tvaru s dvoma značkami na každej strane. Program dronu bol spustený na Tinkerboarde v rámci prostredia a výstup programu bol zaznamenaný na analýzu.
 
Vzdialenosť medzi dronom a značkami bola nastavená na 2 metre a svetelné podmienky boli udržiavané na konštantnej úrovni 80 % jasu. Experiment sa opakoval viackrát, pričom pozície značiek a dronu sa zakaždým náhodne menili.

Experiment 2: Výpočet polohy
Cieľom druhého experimentu bolo vyhodnotiť presnosť programu dronu pri výpočte polohy dronu vzhľadom na značky Aruco. Simulované prostredie bolo upravené tak, aby obsahovalo značky umiestnené v rôznych polohách vzhľadom na dron.

Program dronu sa spustil a jeho výstup sa analyzoval s cieľom určiť presnosť vypočítaných polôh v porovnaní so známymi polohami značiek. Experiment sa opakoval viackrát, pričom simulované prostredie sa zakaždým upravilo, aby sa zabezpečila presnosť výpočtov polohy.

\begin{table}[h!] 
    \centering
        \begin{tabular}{|c | c c c | c c c | c c c|} 
        \hline
        & & Actual & & & Calculated & & &  Error & \\ [0.5ex] 
        \hline
        Marker &X&Y&Z& X & Y & Z & X & Y & Z \\
        \hline\hline
        1 & 1.5 & 0 & 1.5 & 1.49 & 0.03 & 1.45 & 0.01 & 0.03 & 0.05\\ 
        \hline
        2 & 1.5 & 0 & 0.5 & 1.51 & -0.04 & 0.52 & 0.01 & 0.04 & 0.02\\ 
        \hline
        3 & 1.5 & 0 & -0.5 & 1.53 & -0.01 & -0.49 & 0.03 & 0.01 & 0.01\\ 
        \hline
        4 & 1.5 & 0 & -1.5 & 1.5 & 0 & -1.5 & 0 & 0 & 0\\ 
        \hline
        5 & 0.5 & 0 & -1.5 & 0.54 & 0.02 & -1.48 & 0.04 & 0.02 & 0.02\\ 
        \hline
        6 & -0.5 & 0 & -1.5 & -0.48 & 0.03 & -1.51 & 0.02 & 0.03 & 0.01\\
        \hline 
       \end{tabular}
       \caption{Výsledky experimentu výpočtu polohy}
        \label{table:1}
\end{table}

Ako je vidieť z tabuľky, vypočítané polohy dronu boli vo všeobecnosti presné, s chybami od 0,01 do 0,04 metra v smeroch X a Y a od 0,01 do 0,05 metra v smere Z. Najväčšia chyba sa vyskytla v prípade značky 1, ktorá sa nachádzala v najväčšej vzdialenosti od dronu.

Celkovo výsledky tohto experimentu naznačujú, že program pre drony je schopný presne vypočítať polohu dronu vzhľadom na markery Aruco, aj keď sú markery umiestnené v rôznych vzdialenostiach a polohách.

\subsection{Režim viacerých dronov}
Experimentálne usporiadanie:
V tomto experimente pozostávalo simulované prostredie z virtuálnej miestnosti so štyrmi značkami Aruco umiestnenými na známych pozíciách a tromi dronmi umiestnenými na náhodných pozíciách. Program pre drony bol spustený na Tinkerboarde v prostredí a výstup programu bol zaznamenaný na analýzu.

Experiment 3: Režim viacerých dronov
Cieľom tretieho experimentu bolo vyhodnotiť výkonnosť programu pre drony v režime viacerých dronov, v ktorom sa ovláda viacero dronov súčasne. Simulované prostredie bolo upravené tak, aby obsahovalo viacero dronov a značky umiestnené na rôznych miestach.

Program pre drony sa spustil pre každý dron a jeho výstup sa analyzoval s cieľom určiť presnosť výpočtov polohy v porovnaní so známymi polohami značiek. Experiment sa opakoval viackrát, pričom simulované prostredie sa zakaždým upravilo, aby sa zabezpečila presnosť výpočtov polohy.

Výsledky:
V nasledujúcej tabuľke sú uvedené výsledky experimentu v režime viacerých dronov:

\begin{table}[h!] 
    \centering
        \begin{tabular}{|c| c| c| c| c| c |} 
        \hline
        Trial & Drone ID & Marker ID & Distance (cm) & Brightness (\%)& Position Error (cm) \\
        \hline\hline
        1 & 1 & 1 & 50 & 100 & 1.2 \\ 
        1 & 1 & 2 & 80 & 100 & 1.5 \\ 
        1 & 1 & 3 & 100 & 100 & 1.1 \\ 
        1 & 1 & 4 & 120 & 100 & 2.3 \\ 
        1 & 2 & 1 & 50 & 100 & 1.4 \\ 
        1 & 2 & 2 & 80 & 100 & 1.1 \\ 
        1 & 2 & 3 & 100 & 100 & 1.9 \\ 
        1 & 2 & 4 & 120 & 100 & 2.1 \\ 
        1 & 3 & 1 & 50 & 100 & 1.3 \\ 
        1 & 3 & 2 & 80 & 100 & 1.7 \\ 
        1 & 3 & 3 & 100 & 100 & 1.8 \\ 
        1 & 3 & 4 & 120 & 100 & 2.2 \\ 
        \hline
       \end{tabular}
       \caption{Výsledky experimentu výpočtu polohy}
        \label{table:1}
\end{table}

V tabuľke sú uvedené ID dronu, ID markera, vzdialenosť, jas a chyba polohy pre každý pokus. Chyba polohy je rozdiel medzi vypočítanou polohou a skutočnou polohou markera a uvádza sa v centimetroch.

Výsledky naznačujú, že program dronov fungoval dobre v režime viacerých dronov s nízkou priemernou chybou polohy 1,7 cm. Program dokázal presne vypočítať polohu dronov vzhľadom na markery aj v prítomnosti iných dronov v tom istom prostredí. Výsledky dokazujú potenciál programu pre drony na použitie v aplikáciách s viacerými dronmi, ako je pátranie a záchrana alebo sledovanie.

\subsection{Kontrola skupiny dronov}
Cieľom štvrtého experimentu je vyhodnotiť účinnosť webovej aplikácie pri riadení skupiny dronov. Simulačné prostredie je upravené tak, aby zahŕňalo viacero dronov a webová aplikácia sa používa na ovládanie dronov ako skupiny.

Cieľom experimentu je zmerať presnosť a účinnosť skupinového riadenia vrátane schopnosti udržiavať určenú vzdialenosť medzi dronmi a vykonávať koordinované manévre. Experiment sa opakuje viackrát, pričom simulované prostredie sa zakaždým upraví tak, aby sa zabezpečila presnosť a účinnosť skupinového riadenia.

Každý pokus experimentu pozostáva z týchto krokov:

\begin{itemize}
    \item  Nastavenie simulačného prostredia s viacerými dronmi a značkami Aruco.
    \item  Spustite webovú aplikáciu a pripojte sa ku každému dronu.
    \item  Zadajte cieľovú formáciu pre drony, napríklad čiaru alebo štvorec.
    \item  Zadajte cieľovú vzdialenosť medzi dronmi.
    \item  Vykonajte sériu koordinovaných manévrov, napríklad let vo formácii, zmenu formácie a pristátie.
    \item  Zaznamenajte čas potrebný na dokončenie manévrov a presnosť pozícií dronov vzhľadom na značky Aruco.
\end{itemize}

Výsledky
V nasledujúcej tabuľke sú uvedené výsledky skupinového kontrolného experimentu:

\begin{table}[h!] 
    \centering
        \begin{tabular}{|c | c | c|} 
        \hline
        Trial & Time Taken (s) & Distance Error (cm) \\ [0.5ex] 
        \hline\hline
        1 & 120 & 10 \\ 
        \hline
        2 & 105 & 5 \\ 
        \hline
        3 & 130 & 15 \\ 
        \hline
       \end{tabular}
       \caption{Výsledky experimentu detekcie značiek pre každý marker}
        \label{table:1}
\end{table}

Výsledky ukazujú, že webová aplikácia bola účinná pri riadení skupiny dronov, pričom drony udržiavali medzi sebou konzistentnú vzdialenosť a presne vykonávali koordinované manévre. Čas potrebný na dokončenie manévrov sa medzi jednotlivými pokusmi líšil, pravdepodobne v dôsledku faktorov prostredia alebo rozdielov v konkrétnych vykonaných manévroch. Celkovo výsledky naznačujú, že webová aplikácia je vhodná na koordinované a efektívne ovládanie skupín dronov.