\section{Z\'aver}

% Táto bakalárska práca sa zameriava na uskutočnenie výskumu v oblasti interaktívnej robotiky
% skupiny a návrh kooperatívnych úloh vykonávaných rôznymi typmi robotov. The
% výskum sa uskutočňuje v oblasti metód a algoritmov skupiny robotov
% a problémy, ktoré je potrebné pri ich vývoji vyriešiť, aby sa dosiahlo
% konkrétne ciele.
% Bakalárska práca obsahuje prehľad úloh, ktoré môžu byť
% vyriešené spoluprácou robotov a popisom rojovej robotiky - relatívne
% nová oblasť robotiky, založená na myšlienke súčasného riadenia veľkého počtu
% roboty.

Táto bakalárska práca sa zameriava na výskum v oblasti interaktívnych robotických skupín a návrh kooperatívnych úloh vykonávaných rôznymi typmi robotov. Výskum je vedený v oblasti metód a algoritmov riadenia skupiny robotov
a problémy, ktoré je potrebné pri ich vývoji vyriešiť, aby sa dosiahli konkrétne ciele. Bakalárska práca obsahuje prehľad úloh, ktoré je možné vyriešiť spoluprácou robotov, a popis rojovej robotiky - relatívne novej oblasti robotiky, založenej na myšlienke simultánneho riadenia veľkého počtu robotov.
Táto bakalárska práca bola vypracovaná podľa pokynov vedúceho. Podmienečne je možné túto prácu rozdeliť na 4 časti (úlohy), pretože boli vyvinuté výsledky a pre každý režim boli urobené simulácie pre dokonalosť odhadu.
Prvá simulácia bola vykonaná pre riadenie korisťového robota pomocou klávesnice. Boli vytvorené rôzne stopy a odhadovaná efektivita riadenia.
Druhý experiment bol vykonaný pre režim detekčných značiek, ktorý sa používal na detekciu robota. Výsledkom je efektivita detekcie tejto značky v rôznych podmienkach a pod rôznymi uhlami.
Tretia simulácia bola zameraná na formovanie kooperatívnej navigácie. Výsledkom je, že sme presvedčení o dobrej koordinácii medzi robotmi.
A posledná simulácia bola pre proces prenasledovania.
Táto bakalárska práca a uskutočňovanie experimentov sú relevantné, pretože pre spoločnosť je dôležitý aktívny vývoj systémov robotickej spolupráce a potreba skúmania algoritmov a metód riešenia rôznych problémov. Vďaka tomu, že sa ľudia snažia zlepšiť svoju životnú úroveň, bude sa robotika aktívne rozvíjať. V blízkej budúcnosti budú vedci riešiť problémy komunikácie medzi robotmi na veľké vzdialenosti, vytvoria nové spôsoby pohybu a vzájomnej spolupráce pri dosahovaní lepších výsledkov. Tieto experimenty je možné rozšíriť a skomplikovať pridaním nových systémov a typov robotov v mnohých smeroch. Existujú možnosti na vývoj schém a riešení pre nové typy úloh vykonávaných rôznymi typmi robotov alebo na hlboké preskúmanie a vývoj metód pre multi-robotiku.
