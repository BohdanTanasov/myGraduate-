\section{Z\'aver}

% Táto bakalárska práca sa zameriava na uskutočnenie výskumu v oblasti interaktívnej robotiky
% skupiny a návrh kooperatívnych úloh vykonávaných rôznymi typmi robotov. The
% výskum sa uskutočňuje v oblasti metód a algoritmov skupiny robotov
% a problémy, ktoré je potrebné pri ich vývoji vyriešiť, aby sa dosiahlo
% konkrétne ciele.
% Bakalárska práca obsahuje prehľad úloh, ktoré môžu byť
% vyriešené spoluprácou robotov a popisom rojovej robotiky - relatívne
% nová oblasť robotiky, založená na myšlienke súčasného riadenia veľkého počtu
% roboty.

Program pre drony vyvinutý v rámci tohto projektu úspešne preukázal svoju schopnosť ovládať dron, vypočítať jeho polohu vzhľadom na značky Aruco a zobraziť príslušné údaje na používateľsky prívetivom rozhraní. Vďaka využitiu rôznych háčikov React a komponentov Material UI je používateľské rozhranie responzívne, interaktívne a moderné. Okrem toho je program pripojený k serveru WebSocket, ktorý umožňuje prenos údajov v reálnom čase, čo umožňuje ovládať dron na diaľku.

Implementácia detekcie značiek Aruco a výpočtu polohy poskytuje významnú výhodu pri presnom určovaní polohy dronu. Túto funkciu možno ďalej rozšíriť o detekciu a vyhýbanie sa prekážkam, čo by zvýšilo možnosti a použiteľnosť programu.

Záverom možno konštatovať, že tento projekt dosiahol svoje ciele, a to vyvinúť program pre drony, ktorý poskytuje užívateľsky prívetivé ovládanie, výpočet polohy a prenos údajov v reálnom čase. Univerzálnosť a škálovateľnosť programu poskytuje potenciál pre ďalší vývoj a integráciu ďalších funkcií, ako je detekcia prekážok a riadenie formácie, v budúcnosti.