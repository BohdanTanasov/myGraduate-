% !TeX encoding = UTF-8
% !TeX spellcheck = sk_SK
% !TeX root=tukedip.tex

\section*{Úvod}
\addcontentsline{toc}{section}{\numberline{}Úvod}
\setcounter{page}{1}
Bezpilotné lietadlá (UAV) alebo drony, ako sú častejšie známe, sú čoraz bežnejšie, pričom jedným z dôvodov je ich miniaturizácia a nová kategória, ktorú vytvorili: MAV (Micro Aerial Vehicle). Kvadrokoptéry, ktoré sa dajú používať v interiéri, sa vďaka svojim malým rozmerom môžu ľahko presúvať medzi rôznymi typmi zariadení. V širokom spektre aplikácií, ako je vizuálny dohľad, monitorovanie alebo dokonca nahrávanie videa, poskytujú malé rozmery lepšiu manévrovateľnosť a možnosť lietať a skúmať menšie, ťažko dostupné oblasti.

Malá veľkosť je však na úkor menšieho počtu snímačov, keďže hmotnosť je rozhodujúcim faktorom. Jediná zabudovaná kamera so správnym softvérom môže nahradiť mnohé snímače potrebné na vyhýbanie sa prekážkam, čo je dôležité pre autonómnu prevádzku. V súčasnosti ani palubné počítače kompaktných zariadení nemajú takýto výpočtový výkon, takže na spracovanie obrazu a odosielanie riadiacich údajov môžeme použiť externý počítač.

Cieľom tejto práce je ovládať a riadiť dron a viacero dronov pomocou webovej aplikácie a zverejňovať polohu dronu v priestore pomocou kamery a aruko tagov.