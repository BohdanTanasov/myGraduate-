% !TeX encoding = UTF-8
% !TeX spellcheck = sk_SK
% !TeX root=tukedip.tex

\section*{Úvod}
\addcontentsline{toc}{section}{\numberline{}Úvod}
\setcounter{page}{1}
Drony sa v posledných rokoch stávajú čoraz populárnejšími vďaka svojej schopnosti vykonávať širokú škálu úloh, ktoré by inak boli pre človeka náročné alebo nebezpečné. Schopnosť ovládať viacero dronov súčasne otvára ešte viac možností ich využitia. Na dosiahnutie tohto cieľa je potrebné vyvinúť riadiaci systém, ktorý umožní efektívne a intuitívne ovládanie viacerých dronov. Cieľom tejto práce je vyvinúť takýto systém pomocou webovej aplikácie React a socketov.

Navrhovaný riadiaci systém umožní používateľovi ovládať viacero dronov Tello súčasne pomocou webového rozhrania. Použitie značiek Aruco umožní dronom zistiť ich polohu a podľa nej sa navigovať, čo umožní ovládať drony v individuálnom aj skupinovom režime. Systém bude pozostávať z backendu Node.js, ktorý bude zabezpečovať komunikáciu medzi webovou aplikáciou a dronmi, ako aj z programu Python bežiaceho na doske Asus TinkerBoard pripojenej ku každému dronu. Každý dron bude mať špecifické ID, čo umožní ich rozlíšenie.

Vývoj takéhoto riadiaceho systému predstavuje niekoľko výziev vrátane potreby zvládnuť komunikáciu v reálnom čase medzi dronmi a webovou aplikáciou, potreby presne zistiť polohu dronov a potreby zabezpečiť, aby sa systém ľahko používal a poskytoval dobrý používateľský zážitok.

Navrhovaný systém má potenciál na využitie v rôznych aplikáciách, ako je napríklad sledovanie pomocou dronov, letecké fotografovanie alebo pátracie a záchranné operácie. Tým, že systém umožňuje efektívne a intuitívne ovládanie viacerých dronov, by mohol pomôcť zvýšiť bezpečnosť a efektívnosť týchto operácií.

Zvyšok tejto práce je usporiadaný takto. V časti 2 sa uvádza prehľad príslušných technológií a metód, ktoré boli použité pri vývoji systému. V časti 3 je opísaný proces vývoja riadiaceho systému vrátane krokov vykonaných pri návrhu, implementácii a testovaní systému. V časti 4 sú uvedené výsledky práce vrátane podrobného opisu vyvinutého systému, jeho vlastností, používateľského rozhrania a funkčnosti, ako aj ukážky možností systému prostredníctvom príkladov použitia. V časti 5 sa hodnotí výkonnosť a použiteľnosť systému pomocou príslušných metrík a spätnej väzby od používateľov. Nakoniec sa v oddiele 6 uvádza zhrnutie výskumnej otázky, cieľov a prínosov, ako aj diskusia o obmedzeniach systému a možných cestách pre budúcu prácu a zlepšenie.