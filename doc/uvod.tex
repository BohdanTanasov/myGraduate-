% !TeX encoding = UTF-8
% !TeX spellcheck = sk_SK
% !TeX root=tukedip.tex

\section*{Úvod}
\addcontentsline{toc}{section}{\numberline{}Úvod}
\setcounter{page}{1}
Počas posledného desaťročia zaznamenala kooperačná robotika dôležitý vývoj. Systémy s viacerými robotmi sa používajú na rôzne aplikácie na vykonávanie úloh, ktoré sú pre jedného robota náročné alebo časovo náročné. V literatúre sa uvažuje o rôznych aspektoch spolupráce v robotike. Boli navrhnuté rôzne prístupy od Lyapunovovej teórie po teóriu štrukturálnej zložitosti.
Všeobecne sa diskutuje o riadení formovania robota za prítomnosti a neprítomnosti prekážok. Literatúra o formovaní robotov je rozsiahla a navrhujú sa riešenia založené na behaviorálnych a modelových kontrolách. Metódy založené na správaní zvyšujú odozvu v reálnom čase a znižujú zložitosť, ktorá sprevádza presné modely. Tieto metódy však nepodporujú prostriedky na predpovedanie presných výsledkov robotov. Tieto metódy sú známe svojou robustnosťou. 
Sledovanie pohybujúceho sa cieľa pomocou mobilného robota bolo zvážené mnohými výskumníkmi. Nedávno boli na splnenie tej istej úlohy zavedené systémy s viacerými robotmi. Na tento problém sa pozeráme z rôznych uhlov pohľadu a riešime ho pomocou rôznych metód a techník.
V tejto práci bolo navrhnuté sledovanie pohybujúceho sa objektu pomocou kamery na lovcom robotovi a značkovača na koristi. Pomocou techniky, ktorá je v tejto práci napísaná, je možné získať vektor na korisť pre každého robota. Ďalším problémom v tejto časti bola záchrana formácie v procese prenasledovania. Roboty sa pohybujú vo formáciách riadených prostredníctvom formačných vektorov.  A tiež problémom bolo kooperatívne zachytenie cieľa viacerými robotmi, metódy realizácie boli v tejto práci napísané. V poslednej časti boli urobené rôzne skúsenosti a podané výsledky.