% !TeX encoding=utf-8
\documentclass[a4paper]{feidippp}
\usepackage[pdftex]{graphicx}
\DeclareGraphicsExtensions{.pdf,.png.,mps.,jpg}
\graphicspath{{figures/}}

\usepackage[utf8]{inputenc}
\usepackage[T1]{fontenc}
\usepackage{listings}
\usepackage{color}

\definecolor{codegreen}{rgb}{0,0.6,0}
\definecolor{codegray}{rgb}{0.5,0.5,0.5}
\definecolor{codepurple}{rgb}{0.58,0,0.82}
\definecolor{backcolour}{rgb}{0.95,0.95,0.92}

\lstdefinestyle{mystyle}{
    backgroundcolor=\color{backcolour},   
    commentstyle=\color{codegreen},
    keywordstyle=\color{magenta},
    numberstyle=\tiny\color{codegray},
    stringstyle=\color{codepurple},
    basicstyle=\footnotesize,
    breakatwhitespace=false,         
    breaklines=true,                 
    captionpos=b,                    
    keepspaces=true,                 
    numbers=left,                    
    numbersep=5pt,                  
    showspaces=false,                
    showstringspaces=false,
    showtabs=false,                  
    tabsize=2
}

\lstset{style=mystyle}


\usepackage[slovak]{babel}



\usepackage{lmodern}

\usepackage[document]{ragged2e}

\usepackage{amsmath,amssymb,amsfonts}

\usepackage{listings}

\usepackage{indentfirst}
\usepackage{parskip}
\setlength{\parskip}{1em}

\def\figurename{Obrázok}
\def\tabname{Tabuľka}
 


%\usepackage[dvips]{graphicx}
%\DeclareGraphicsExtensions{.eps}

\usepackage[edges]{forest}

\definecolor{foldercolor}{RGB}{124,166,198}

\tikzset{pics/folder/.style={code={%
    \node[inner sep=0pt, minimum size=#1](-foldericon){};
    \node[folder style, inner sep=0pt, minimum width=0.3*#1, minimum height=0.6*#1, above right, xshift=0.05*#1] at (-foldericon.west){};
    \node[folder style, inner sep=0pt, minimum size=#1] at (-foldericon.center){};}
    },
    pics/folder/.default={20pt},
    folder style/.style={draw=foldercolor!80!black,top color=foldercolor!40,bottom color=foldercolor}
}

\forestset{is file/.style={edge path'/.expanded={%
        ([xshift=\forestregister{folder indent}]!u.parent anchor) |- (.child anchor)},
        inner sep=1pt},
    this folder size/.style={edge path'/.expanded={%
        ([xshift=\forestregister{folder indent}]!u.parent anchor) |- (.child anchor) pic[solid]{folder=#1}}, inner xsep=0.6*#1},
    folder tree indent/.style={before computing xy={l=#1}},
    folder icons/.style={folder, this folder size=#1, folder tree indent=3*#1},
    folder icons/.default={12pt},
}

%\usepackage[pdftex]{hyperref} %% tlac !!!
\usepackage[pdftex,colorlinks,citecolor=magenta,bookmarksnumbered,unicode,pdftoolbar=true,pdfmenubar=true,pdfwindowui=true,bookmarksopen=true]{hyperref}
\hypersetup{%
baseurl={http://www.tuke.sk/sevcovic},
pdfcreator={pdfcsLaTeX},
pdfkeywords={Používateľská priručka},
pdftitle={Simulácia kooperácie multi-robotickéhosystému},
pdfauthor={Bohdan Tanasov},
pdfsubject={Používateľská priručka}
}

% Citovanie podla mena autora a roku
%\usepackage[numbers]{natbib}
\usepackage{natbib} \citestyle{chicago}

%\usepackage{mathptm} %\usepackage{times}

\katedra{Umelej inteligencie}
\department{Fakulta elektrotechniky a informatiky}
\odbor{Inteligentné systémy}
\autor{Bohdan Tanasov}
\veduci{doc. Dr. Ing.~Ján~Vaščák}
\konzultant{ing.~Dušan~Herich}
\nazov{Simulation of Cooperation for a Multi-Robotic System}
% \kratkynazov{Optimalizácia písania DP}
\nazovprogramu{Ovládanie viacerých dronov pomocou webu}
\klucoveslova{optimalizácia, diplomová práca, písanie}
\title{The optimization of the diploma writing at our faculty}
\keywords{optimization, diploma, writing}
\datum{28. 5. 2023}



\begin{document}
\bibliographystyle{dcu}



\titulnastrana

\newpage


\tableofcontents

\newpage

\addcontentsline{toc}{section}{\numberline{}Zoznam obrázkov}
\listoffigures

\newpage


\setcounter{page}{1}

\section{Funkcia programu}
Vitajte v používateľskej príručke pre systém riadenia dronov Tello, ktorý bol vyvinutý v rámci magisterského projektu. Tento systém je určený na ovládanie viacerých dronov Tello pomocou webového rozhrania vytvoreného pomocou React a socketov. Pomocou značkovačov Aruco sú drony schopné zisťovať svoju polohu a podľa nej sa navigovať.

Systém ovládania dronov Tello ponúka dva režimy prevádzky, individuálny a skupinový. V individuálnom režime môže používateľ ovládať jeden dron, zatiaľ čo v skupinovom režime môže vidieť stav všetkých pripojených dronov, ale môže ovládať vždy len jeden. Webové rozhranie poskytuje tlačidlo na prepínanie medzi týmito dvoma režimami spolu so zoznamom ikon predstavujúcich pripojené drony. Používateľ môže kliknutím na ikonu aktivovať dron v individuálnom režime a stav dronu sa zobrazí v skupinovom režime.

Systém ovládania dronov Tello poskytuje používateľovi aj rôzne ovládacie prvky na ovládanie dronov. Medzi ovládacie prvky patria: dopredu, doľava, dozadu, doprava, vzlet, pristátie, hore, dole, otočenie doľava a otočenie doprava. Tieto ovládacie prvky sú k dispozícii v individuálnom aj skupinovom režime.

Táto používateľská príručka vás prevedie inštaláciou a prevádzkou systému ovládania dronov Tello. Či už ste začiatočník alebo skúsený používateľ, táto príručka vám poskytne všetky potrebné informácie na efektívne ovládanie systému ovládania dronu Tello.

\newpage
\section{Začíname}
Skôr ako budete môcť používať riadiaci systém pre viacero dronov Tello pomocou webového rozhrania, musíte sa uistiť, že máte potrebný hardvér a softvér.

\subsection{Požiadavky na hardvér}
Na používanie riadiaceho systému budete potrebovať nasledujúci hardvér:
\begin{itemize}
    \item Viacero dronov Tello: Presný počet potrebných dronov bude závisieť od konkrétneho prípadu použitia.
    \item Asus TinkerBoard: Toto je centrálny ovládač, ktorý sa bude používať na spúšťanie programov v jazyku Python pre každý dron.
    \item Značkovače Aruco: Tie sa budú používať na presné určenie polohy dronov.
\end{itemize}

\subsection{Požiadavky na softvér}
Okrem požadovaného hardvéru je potrebné mať nainštalovaný aj nasledujúci softvér:

\begin{itemize}
  \item Webový prehliadač: Webové rozhranie je prístupné pomocou akéhokoľvek moderného webového prehliadača, napríklad Google Chrome, Mozilla Firefox alebo Microsoft Edge.
  \item Node.js: Ide o runtime jazyka JavaScript, ktorý je potrebný na spustenie backendu riadiaceho systému.
  \item Python: Programy pre drony budú napísané v jazyku Python, preto je potrebné, aby ste mali na Asus TinkerBoard nainštalovaný jazyk Python.
\end{itemize}

\subsection{Pripojenie hardvéru}
Aby ste mohli začať pracovať s riadiacim systémom, budete musieť pripojiť hardvér nasledujúcim spôsobom:

\begin{itemize}
  \item Pripojte každý dron Tello k doske Asus TinkerBoard pomocou pripojenia Wi-Fi.
  \item Namontujte značky Aruco na vhodné miesta v pracovnej oblasti dronu, aby ste zabezpečili presné určenie polohy.
\end{itemize}

\subsection{Spustenie riadiaceho systému}
Keď máte nainštalovaný a pripojený všetok potrebný hardvér a softvér, môžete spustiť riadiaci systém podľa nasledujúcich krokov:

\begin{itemize}
  \item Klonujte repozitár projektu z GitHub.
  \item Nainštalujte závislosti Node.js spustením príkazu npm install v adresári projektu.
  \item Pripojte Asus TinkerBoard k zdroju napájania a spustite ho.
  \item Spustite program Python pre každý dron na Asus TinkerBoard.
  \item Spustite server Node.js spustením príkazu node index.js v adresári projektu.
  \item Otvorte webový prehliadač a prejdite na adresu URL webového rozhrania.
  \item Podľa pokynov na obrazovke ovládajte drony jednotlivo alebo v skupinovom režime.
  \item Gratulujeme, teraz ste úspešne spustili systém ovládania viacerých dronov Tello pomocou webového rozhrania!
\end{itemize}

\newpage
\section{Individuálne a skupinové režimy}
Systém ovládania viacerých dronov Tello umožňuje používateľovi prepínať medzi dvoma rôznymi režimami: Individuálny a skupinový.

\subsection{Individuálny režim}
V individuálnom režime môže používateľ naraz ovládať jeden dron. Ak chcete vybrať dron, používateľ môže jednoducho kliknúť na príslušnú ikonu/tlačidlo dronu zo zoznamu pripojených dronov zobrazeného vo webovom rozhraní. Po výbere dronu môže používateľ na ovládanie pohybu dronu použiť dostupné ovládacie prvky, ako sú pohyb dopredu, dozadu, doľava, doprava, nahor, nadol a preklopenie doľava/doprava. Používateľ môže tiež pristáť alebo vzlietnuť s dronom pomocou príslušných tlačidiel.

\subsection{Režim skupiny}
V skupinovom režime môže používateľ zobraziť stav všetkých pripojených dronov súčasne, ale nemôže ovládať žiadny jednotlivý dron. Zobrazí sa zoznam pripojených dronov spolu s ich stavmi, napríklad či práve letia, pristávajú alebo sa nabíjajú. Používateľ môže zobraziť živý videokanál každého dronu kliknutím na príslušnú ikonu/tlačidlo dronu zo zoznamu. Okrem toho môže používateľ použiť dostupné ovládacie prvky, ako je priblíženie a oddialenie, na úpravu videokamery na lepšie sledovanie.

\subsection{Prepínacie tlačidlo}
Používateľ môže prepínať medzi individuálnym a skupinovým režimom kliknutím na prepínacie tlačidlo umiestnené vo webovom rozhraní. Keď používateľ prepne do individuálneho režimu, zobrazí sa zoznam pripojených dronov spolu s ich stavmi a ikonami/tlačidlami na ich ovládanie. Keď používateľ prepne do režimu skupiny, zobrazí sa zoznam pripojených dronov spolu s ich stavmi a ikonami/tlačidlami na zobrazenie ich živých videokanálov.

Poznámka: Je dôležité zabezpečiť, aby dron ovládal vždy len jeden používateľ, aby sa predišlo konfliktom a kolíziám medzi dronmi. Preto bol riadiaci systém navrhnutý tak, aby v individuálnom režime mohol byť v danom čase len jeden používateľ. Ak sa iný používateľ pokúsi vstúpiť do individuálneho režimu, keď už dron ovláda iný používateľ, zobrazí sa chybové hlásenie a nebude môcť ovládať dron, kým ho aktuálny používateľ neuvoľní.

\newpage
\section{Ovládanie dronov}

Keď je dron vybraný v individuálnom režime, používateľ má plnú kontrolu nad pohybom a správaním tohto dronu. K dispozícii sú nasledujúce ovládacie prvky:

\begin{itemize}
\item Vzlet/Pristátie: Tlačidlo Vzlet zdvihne dron zo zeme a vznáša sa v pevnej výške, zatiaľ čo tlačidlo Pristátie vráti dron späť na zem a vypne jeho motory.
\item Dopredu/dozadu: Tlačidlá Vpred a Vzad posúvajú dron dopredu, resp. dozadu. Prejdená vzdialenosť závisí od dĺžky podržania tlačidla.
\item Vľavo/vpravo: Tlačidlá Vľavo a Vpravo posúvajú dron doľava, resp. doprava. Prejdená vzdialenosť závisí od toho, ako dlho je tlačidlo podržané.
\item Nahor/Dole: Tlačidlami Nahor a Nadol sa dron pohybuje nahor, resp. nadol. Prejdená vzdialenosť závisí od toho, ako dlho je tlačidlo podržané.
\item Preklopenie doľava/doprava: Tlačidlá Flip Left (Prevrátiť doľava) a Flip Right (Prevrátiť doprava) vykonajú otočenie o 360 stupňov doľava, resp. doprava.
\item Núdzové zastavenie: Tlačidlo núdzového zastavenia okamžite preruší napájanie motorov dronu a spôsobí jeho pád z oblohy. Toto tlačidlo by sa malo používať len v prípade núdze.
\end{itemize}

V režime skupiny môže používateľ stále ovládať drony jednotlivo, ale môže tiež vydávať príkazy všetkým dronom v skupine naraz. K dispozícii sú nasledujúce ovládacie prvky:

\begin{itemize}
\item Prepínanie skupinového režimu: Tlačidlo Prepnúť skupinový režim prepína rozhranie medzi individuálnym a skupinovým režimom.
\item Vybrať/zrušiť výber dronov: Na obrazovke sa zobrazí zoznam dostupných dronov a používateľ môže vybrať jeden alebo viac dronov, ktoré chce ovládať. Vybrané drony sa zvýraznia a ich stavy sa zobrazia v časti Drone Status (Stav dronu).
\item Vzlet všetkých/Pristátie všetkých: Tlačidlo Vzlet všetkých zdvihne všetky drony v skupine nad zem, aby sa vznášali v pevnej výške, zatiaľ čo tlačidlo Pristátie všetkých vráti všetky drony späť na zem a vypne ich motory.
\item Dopredu/dozadu všetky: Tlačidlá Dopredu všetky a Dozadu všetky posúvajú všetky drony v skupine dopredu, resp. dozadu.
\item Vľavo/vpravo všetko: Tlačidlá Vľavo všetko a Vpravo všetko posúvajú všetky drony v skupine doľava, resp. doprava.
\item Nahor/Dole Všetky: Tlačidlá Nahor Všetky a Nadol Všetky posúvajú všetky drony v skupine nahor, resp. nadol.
\item Flip Left/Right All (Prevrátiť doľava/doprava všetky): Tlačidlá Flip Left All (Prevrátiť doľava všetky) a Flip Right All (Prevrátiť doprava všetky) vykonajú pre všetky drony v skupine otočenie o 360 stupňov doľava alebo doprava.
\item Emergency Stop All (Núdzové zastavenie všetkých): Tlačidlo Emergency Stop All (Núdzové zastavenie všetkých) okamžite preruší napájanie motorov všetkých dronov v skupine a spôsobí ich pád z oblohy. Toto tlačidlo by sa malo používať len v prípade núdze.
\end{itemize}

Časť Stav dronu zobrazuje stav každého dronu v skupine vrátane úrovne nabitia batérie, nadmorskej výšky a toho, či práve letí alebo je na zemi. Používateľ môže tieto informácie použiť na prijímanie informovaných rozhodnutí o spôsobe ovládania dronov.

\newpage
\section{Riešenie problémov}

Napriek tomu, že sme urobili všetko pre to, aby bol náš systém užívateľsky prívetivý a ľahko použiteľný, môžu sa vyskytnúť problémy, na ktoré narazíte. Tu sú uvedené niektoré bežné problémy a spôsob ich riešenia:

\subsection{Problémy s pripojením dronu}

\begin{itemize}
\item Ak máte problémy s pripojením k dronu, uistite sa, že je zapnutý a jeho batéria je nabitá. Tiež sa uistite, že je v dosahu TinkerBoard a že je pripojený k rovnakej sieti WiFi.
\item Ak máte stále problémy, skúste resetovať dron a/alebo TinkerBoard.
\end{itemize}

\subsection{Problémy s detekciou značky Aruco}

\begin{itemize}
\item Ak majú drony problémy s detekciou svojej polohy pomocou značiek Aruco, skontrolujte, či sú značky správne umiestnené a či sa nachádzajú v zornom poli kamery. Uistite sa tiež, že sú vhodné svetelné podmienky.
\item Ak problém pretrváva, skúste upraviť nastavenia kamery alebo použiť iný typ markera.
\end{itemize}

\subsection{Problémy s webovou aplikáciou}

\begin{itemize}
\item Ak sa webová aplikácia nenačíta alebo sa správa nepravidelne, skúste obnoviť stránku alebo vymazať vyrovnávaciu pamäť prehliadača.
\item Ak problém pretrváva, skúste použiť iný webový prehliadač alebo zariadenie.
\end{itemize}

\subsection{Problémy s ovládaním}

\begin{itemize}
\item Ak máte problémy s ovládaním dronov, uistite sa, že ste v správnom režime (individuálny alebo skupinový) a že je vybraný dron, ktorý chcete ovládať.
\item Ak problém pretrváva, skúste reštartovať TinkerBoard alebo resetovať drony.
\item Ak žiadne z týchto riešení nepomôže, obráťte sa na náš tím podpory, ktorý vám poskytne ďalšiu pomoc.
\end{itemize}

\newpage
\section{Zaver}

Na záver možno konštatovať, že systém ovládania viacerých dronov Tello pomocou webovej aplikácie React a soketov bol úspešne vyvinutý. Systém využíva značky Aruco, ktoré umožňujú dronom zisťovať ich polohu a podľa toho sa navigovať. Webové rozhranie umožňuje používateľovi ovládať drony v individuálnom aj skupinovom režime s možnosťou prepínania medzi týmito dvoma režimami.

V individuálnom režime môže používateľ ovládať vybraný dron pomocou rôznych ovládacích prvkov, ako napríklad dopredu, doľava, dozadu, doprava, pristáť, vzlietnuť, hore, dole, otočiť sa doľava, otočiť sa doprava a podobne. V skupinovom režime môže používateľ vidieť stav všetkých pripojených dronov v zozname ikon a prepínať medzi nimi na zobrazenie ich konkrétneho stavu.

O komunikáciu medzi webovou aplikáciou a dronmi sa stará backend Node.js a každý dron je pripojený k doske Asus TinkerBoard, na ktorej beží program v jazyku Python s prideleným špecifickým ID. Systém bol dôkladne otestovaný a ukázalo sa, že funguje podľa očakávaní.

Celkovo možno povedať, že vyvinutý systém poskytuje efektívny a používateľsky prívetivý spôsob ovládania viacerých dronov Tello súčasne. Má potenciál uplatniť sa v rôznych odvetviach, ako je kinematografia, poľnohospodárstvo, pátracia a záchranná služba a ďalšie, kde sa vyžaduje spolupráca viacerých dronov pri plnení zložitých úloh.

% \section{Popis vstupných, výstupných a~pracovných súborov} 

% Napr. formáty súborov, atd.

% \section{Obmedzenia programu}

% Zoznam obmedzení programu.

% \section{Chybové hlásenia}

% Chybové hlásenia vlastné pre daný aplikacný program.

% Chybové hlásenia súvisiace s~OS, Štandard. p.~p., \dots).

% \section{Príklad použitia}

% Príklad použitia programu.

% \section{Popis demo-verzie}

% Popis demo-verzie. V~prípade, že navrhovaný program je demo-verzia.


% \addcontentsline{toc}{section}{\numberline{}Zoznam tabuli}
% \listoftables


% \def\refname{Zoznam použitej literatúry}
% \addcontentsline{toc}{section}{\numberline{}Zoznam použitej literatúry}

% \begin{thebibliography}{999}
% \harvarditem{Barančok et al.}{1995}{barancok}
% BARANČOK, D. et al. 1995. \emph{The effect of semiconductor surface treatment on LB film/Si interface.} In:~Physica Status Solidi /a/,  ISSN 0031-8965, 1995, vol. 108, no.2, pp. K~87\,--\,90

% \harvarditem{Gonda}{2001}{gonda}
% GONDA, V. 2001. \emph{Ako napísať a~úspešne obhájiť diplomovú prácu.} Bratislava : Elita, 2001, 3.~doplnené a~prepracované vydanie, 120~s. ISBN 80-8044-075-1

% \harvarditem{Jadr. fyz. a~tech.}{1985}{slovnik}
% \emph{Jadrová fyzika a~technika: Terminologický výkladový slovník.} 2.~rev.~vyd. Bratislava : ALFA, 1985. 235 s. ISBN 80-8256-030-5

% \harvarditem{Katuščák}{1998}{kat}
% KATUŠČÁK, D. \emph{Ako písať vysokoškolské a~kvalifikačné práce.} Bratislava : Stimul, 1998, 2.~doplnené vydanie. 121~s. ISBN 80-85697-82-3

% %\harvarditem{Sýkora a~i.}{1980}{sykora}
% %SÝKORA, F. a~iní. 1980. \emph{Telesná výchova a~šport.} 1.~vyd. Bratislava : SPN, 1980. 35 s. ISBN 80-8046-020-5

% \harvarditem{Lamoš a~Potocký}{1989}{lamos}
% LAMOŠ, F. -- POTOCKÝ, R. 1989. \emph{Pravdepodobnosť a~matematická štatistika.} 1.~vyd. Bratislava : Alfa, 1989. 344 s. ISBN 80-8046-020-5

% %\harvarditem{Sýkora a~i.}{1980}{sykora}
% %SÝKORA, F. a~iní. 1980. \emph{Telesná výchova a~šport.} 1.~vyd. Bratislava : SPN, 1980. 35 s. ISBN 80-8046-020-5

% \harvarditem{Lamoš a~Potocký}{1989}{lamos}
% LAMOŠ, F. -- POTOCKÝ, R. 1989. \emph{Pravdepodobnosť a~matematická štatistika.} 1.~vyd. Bratislava : Alfa, 1989. 344 s. ISBN 80-8046-020-5


% \end{thebibliography}


\end{document}