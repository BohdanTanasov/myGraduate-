% !TeX encoding=utf-8
\documentclass[a4paper]{feidipsp}
\usepackage[pdftex]{graphicx}
\DeclareGraphicsExtensions{.pdf}
\graphicspath{{figures/}}

\usepackage[slovak]{babel}
\usepackage{listings}  
\usepackage{color}
 
\definecolor{codegreen}{rgb}{0,0.6,0}
\definecolor{codegray}{rgb}{0.5,0.5,0.5}
\definecolor{codepurple}{rgb}{0.58,0,0.82}
\definecolor{backcolour}{rgb}{0.95,0.95,0.92}
 
\lstdefinestyle{mystyle}{
    backgroundcolor=\color{backcolour},   
    commentstyle=\color{codegreen},
    keywordstyle=\color{magenta},
    numberstyle=\tiny\color{codegray},
    stringstyle=\color{codepurple},
    basicstyle=\footnotesize,
    breakatwhitespace=false,         
    breaklines=true,                 
    captionpos=b,                    
    keepspaces=true,                 
    numbers=left,                    
    numbersep=5pt,                  
    showspaces=false,                
    showstringspaces=false,
    showtabs=false,                  
    tabsize=2
}

\lstset{style=mystyle}

\usepackage[utf8]{inputenc}
\usepackage[T1]{fontenc}
\usepackage{lmodern}
\usepackage[document]{ragged2e}

\usepackage{indentfirst}
\usepackage{parskip}
\setlength{\parskip}{1em}

\usepackage{amsmath,amsfonts,amssymb,latexsym}

\usepackage{listings}
\usepackage{pythonhighlight}
\lstnewenvironment{mypython}[1][]{\lstset{style=mypython,#1}}{}
 
\def\figurename{Obrázok}  
\def\tabname{Tabuľka}

%\usepackage[dvips]{graphicx}
%\DeclareGraphicsExtensions{.eps}

%\usepackage[pdftex]{hyperref}   %% tlac !!!
\usepackage[pdftex,colorlinks,citecolor=magenta,bookmarksnumbered,unicode,pdftoolbar=true,pdfmenubar=true,pdfwindowui=true,bookmarksopen=true]{hyperref}
\hypersetup{%
baseurl={http://www.tuke.sk/sevcovic},
pdfcreator={pdfcsLaTeX},
pdfkeywords={Sytémová príručka},
pdftitle={Šablóna na písanie DP na FEI TU v~Košiciach},
pdfauthor={Ján Buša, Ladislav Ševčovič},
pdfsubject={Ako napísať peknú DP}
}

% Citovanie podla mena autora a roku
%\usepackage[numbers]{natbib}
\usepackage{natbib} \citestyle{chicago}


\usepackage[edges]{forest}
\usepackage[edges]{forest}

\definecolor{foldercolor}{RGB}{124,166,198}

\tikzset{pics/folder/.style={code={%
    \node[inner sep=0pt, minimum size=#1](-foldericon){};
    \node[folder style, inner sep=0pt, minimum width=0.3*#1, minimum height=0.6*#1, above right, xshift=0.05*#1] at (-foldericon.west){};
    \node[folder style, inner sep=0pt, minimum size=#1] at (-foldericon.center){};}
    },
    pics/folder/.default={20pt},
    folder style/.style={draw=foldercolor!80!black,top color=foldercolor!40,bottom color=foldercolor}
}

\forestset{is file/.style={edge path'/.expanded={%
        ([xshift=\forestregister{folder indent}]!u.parent anchor) |- (.child anchor)},
        inner sep=1pt},
    this folder size/.style={edge path'/.expanded={%
        ([xshift=\forestregister{folder indent}]!u.parent anchor) |- (.child anchor) pic[solid]{folder=#1}}, inner xsep=0.6*#1},
    folder tree indent/.style={before computing xy={l=#1}},
    folder icons/.style={folder, this folder size=#1, folder tree indent=3*#1},
    folder icons/.default={12pt},
}

%\usepackage{mathptm} %\usepackage{times}

\katedra{Umelej inteligencie}
\department{Fakulta elektrotechniky a informatiky}
\odbor{Inteligentné systémy}
\autor{Bohdan Tanasov}
\veduci{doc. Dr. Ing.~Ján~Vaščák}
\konzultant{ing.~Dušan~Herich}
\nazov{Simulation of Cooperation for a Multi-Robotic System}
% \kratkynazov{Optimalizácia písania DP}
\nazovprogramu{Ovládanie viacerých dronov pomocou webu}
\klucoveslova{optimalizácia, diplomová práca, písanie}
\title{The optimization of the diploma writing at our faculty}
\keywords{optimization, diploma, writing}
\datum{28. 5. 2023}



\begin{document}
\bibliographystyle{dcu}

\titulnastrana

\tableofcontents

\newpage

\setcounter{page}{1}

\section{Funkcia programu}

Cieľom tohto projektu je riadiaci systém pre viacero dronov Tello. Systém bude využívať aplikáciu WEB React a sockety, aby poskytol užívateľsky prívetivé rozhranie na ovládanie dronov. Budú sa využívať značky Aruco, ktoré umožnia dronom zistiť ich polohu a podľa toho sa navigovať.

Systém bude pozostávať z webového rozhrania, ktoré poskytne individuálny aj skupinový režim ovládania dronov. Používateľ bude môcť ovládať každý dron jednotlivo alebo ako skupinu a vykonávať zložité manévre a formácie.

Na dosiahnutie tohto cieľa bude projekt využívať backend Node.js, ktorý bude zabezpečovať komunikáciu medzi webovou aplikáciou a dronmi. To umožní ovládanie a monitorovanie dronov v reálnom čase. Backend bude tiež poskytovať možnosť prijímať a ukladať údaje z dronov, ako je úroveň batérie, nadmorská výška a súradnice GPS.

Okrem toho bude systém pripojený k informačnému panelu Asus Dashboard, ktorý sa bude používať na monitorovanie a ovládanie dronov. Na prístrojovej doske sa budú zobrazovať údaje z dronov v reálnom čase, napríklad živé videoprenosy a telemetrické údaje. Prístrojová doska bude tiež poskytovať možnosť ovládať drony, napríklad spúšťať a zastavovať motory a upravovať výšku a orientáciu dronov.

Celkovo bude systém poskytovať komplexné riešenie na ovládanie a monitorovanie viacerých dronov Tello v rôznych scenároch. Vďaka možnosti ovládať každý dron samostatne alebo ako skupinu, využívajúc značky Aruco na navigáciu a poskytujúc údaje a kontrolu v reálnom čase prostredníctvom webovej aplikácie a ovládacieho panela, tento systém výrazne rozšíri možnosti dronov Tello.

\section{Analýza riešenia}

Riadiaci systém pre viacero dronov Tello je navrhnutý tak, aby pozostával z niekoľkých vzájomne prepojených softvérových komponentov. Nasledujúce podkapitoly poskytujú prehľad funkcií a interakcií jednotlivých komponentov na vysokej úrovni.

\subsection{Webová aplikácia React}
Webová aplikácia React App slúži ako primárne používateľské rozhranie pre riadiaci systém. Poskytuje používateľsky prívetivý ovládací panel, ktorý zobrazuje stav dronov, umožňuje používateľovi ovládať jednotlivé drony a umožňuje skupinové ovládanie dronov. Aplikácia používa webové sokety na komunikáciu s backendom Node.js, ktorý riadi komunikáciu dronov.

\subsection{Backend Node.js}
Backend Node.js funguje ako most medzi webovou aplikáciou React a dronmi. Prijíma príkazy z aplikácie a posiela ich dronom prostredníctvom SDK dronov. Taktiež prijíma telemetrické údaje z dronov a posiela ich späť do aplikácie na zobrazenie. Backend používa webové sokety, ktoré umožňujú komunikáciu s aplikáciou v reálnom čase.

\subsection{Detekcia značiek Aruco}
Systém detekcie značiek Aruco je zodpovedný za zisťovanie polohy každého dronu v reálnom čase. Každý dron je vybavený kamerou, ktorá dokáže rozpoznať značky Aruco umiestnené na zemi. Značky slúžia ako referenčný bod pre každý dron a umožňujú dronu určiť svoju polohu a orientáciu v 3D priestore. Systém detekcie značiek je implementovaný pomocou OpenCV, knižnice počítačového videnia s otvoreným zdrojovým kódom.

\subsection{Asus Dashboard}
Asus Dashboard je hardvérový komponent riadiaceho systému, ktorý poskytuje ďalšie možnosti monitorovania a ovládania. Je pripojený k dronom prostredníctvom Wi-Fi a dokáže zobrazovať telemetrické údaje dronu vrátane úrovne nabitia batérie, nadmorskej výšky a rýchlosti. Prístrojovú dosku možno použiť aj na manuálne ovládanie dronov v prípade zlyhania primárneho riadiaceho systému.

Celkovo je architektúra riadiaceho systému navrhnutá ako modulárna, škálovateľná a flexibilná, čo umožňuje jednoduchú integráciu ďalších komponentov a prispôsobenie pre rôzne prípady použitia.

\section{Inštalácia, konfigurácia a používanie}
\subsection{Požiadavky}

Pred inštaláciou a používaním riadiaceho systému pre viacero dronov Tello sa uistite, že máte nasledujúce požiadavky:
\begin{itemize}
    \item Node.js (verzia 14 alebo novšia)
    \item Python (verzia 3.6 alebo novšia)
    \item Drony Tello (viacero)
\end{itemize}
    
\subsection{Installation}
Ak chcete nainštalovať systém, postupujte podľa nasledujúcich krokov:

\begin{enumerate}
    \item Nainštalujte Node.js a npm do svojho počítača.
    \item Naklonujte repozitár GitHub obsahujúci kód systému.
    \item Nainštalujte požadované balíky Node.js spustením nasledujúceho príkazu z koreňového adresára projektu:
        \begin{verbatim}
            npm install
        \end{verbatim}
    \item Nainštalujte Python 3.6 alebo novšiu verziu na každý ovládací panel, ktorý sa bude používať na ovládanie dronov:    
    \begin{enumerate}
        \item Nainštalujte Python 3.x: Program vyžaduje, aby bol na palubnej doske a počítačoch dronu nainštalovaný Python 3.x. Najnovšiu verziu Pythonu si môžete stiahnuť a nainštalovať z oficiálnej webovej stránky Pythonu.
        \item Nainštalujte požadované knižnice: Program vyžaduje inštaláciu niekoľkých knižníc:
        
        \begin{verbatim}
            OpenCV, 
            NumPy, 
            SocketIO, 
            djitellopy,
            pygame,
            opencv-python 4.1.0,
            opencv-contrib 4.1.0,
            numpy,
            scipy,
            pykalman,
            matplotlib
        \end{verbatim}

        Tieto knižnice môžete nainštalovať pomocou správcu balíkov pip spustením nasledujúceho príkazu v termináli:
        \begin{verbatim}
            pip install ...
        \end{verbatim}

        \item Spustite program: Ak chcete spustiť program Python, prejdite do adresára programu a v termináli spustite nasledujúci príkaz:        
        \begin{verbatim}
            python main.py
        \end{verbatim}
        Program sa pripojí k palubnej doske a počítačom dronu a začne odosielať a prijímať údaje.
        \item Použite systém: Po spustení programu Python môžete pomocou webového rozhrania ovládať drony v individuálnom aj skupinovom režime. Drony môžete monitorovať a ovládať aj pomocou prístrojovej dosky Asus.

    \end{enumerate}
\end{enumerate}

\subsection{Konfigurácia}

\begin{enumerate}
    \item Uistite sa, že sú drony nabité a zapnuté.
    \item Pred spustením aplikácie zabezpečte, aby bol prístroj na ovládacom paneli pripojený k sieti Wi-Fi dronu, ktorý bude ovládať. Je to potrebné na nadviazanie spojenia medzi prístrojovým panelom a dronom.
\end{enumerate}

\subsection{Použitie}
Ak chcete používať systém, postupujte podľa nasledujúcich krokov:

\begin{enumerate}
    \item Spustite backend Node.js spustením nasledujúceho príkazu z koreňového adresára projektu:
    
    \begin{verbatim}
    npm start
    \end{verbatim}

    \item Spustite program dashboard na každom počítači s dashboardom spustením nasledujúceho príkazu:
    
\begin{verbatim}
python dashboard.py
\end{verbatim}
 
\item Otvorte webový prehliadač a prejdite na adresu URL webovej aplikácie. Zvyčajne to bude \texttt{http://<server\_ip>:3000}.
\item Použite webové rozhranie na ovládanie dronov v individuálnom alebo skupinovom režime.
\end{enumerate}

\section{Popis hlavnych modulov}

Kameru dronu musíte tiež vopred nakalibrovať pomocou kalibračného algoritmu šachovnice OpenCV. V súbore cam\_class.py môžete nastaviť viacero parametrov pre dĺžku okraja dlaždice šachovnice na kalibráciu, dĺžku okraja značky na meranie a filtrovanie okrajov pre skreslené značky.

Pôvodný skript djitellopy/tello.py som upravil a doplnil o čítanie stavu Tello, ktoré prebieha v samostatnom vlákne a neblokuje hlavné vykonávanie.

Pre let je potrebné spustiť súbor main.py.

Všetky transformácie matíc sa vykonávajú pomocou funkcií v transformations.py. Základný princíp spočíva v tom, že keď sú dve značky viditeľné súčasne, môžete vypočítať transformačnú maticu medzi týmito dvoma súradnicovými systémami. (Užitie viacerých vzoriek na spriemerovanie matíc medzi dvoma markermi.) Dron môže takto mapovať svoju cestu pomocou reťazca transformácií z globálneho počiatku, pričom ako základ sa použije prvý videný marker. Všetky transformácie súradníc sa vykonávajú v reálnom čase, globálne body sa potom uložia.

\section{Riešenie problémov}

Hoci sme venovali veľkú pozornosť tomu, aby bol náš systém čo najspoľahlivejší a bezchybný, pri jeho používaní sa môžete stretnúť s niektorými problémami. V tejto časti vám poskytneme niekoľko pokynov na riešenie bežných problémov, s ktorými sa môžete stretnúť.

\subsection{Drony sa nepripájajú k systému}

Ak máte problémy s pripojením dronov k systému, uistite sa, že sú správne zapnuté a pripojené k tej istej sieti WiFi ako ovládací panel. Tiež sa uistite, že IP adresa dronu je správne nastavená v konfiguračnom súbore.

\subsection{Prístrojová doska nezobrazuje pozície dronov}

Ak sa na prístrojovej doske nezobrazujú polohy dronov, skontrolujte, či sú značky Aruco správne umiestnené a v zornom poli kamier dronov. Taktiež sa uistite, že sa do prístrojového panela správne prenášajú údaje z kamery.

\subsection{Systém nereaguje alebo padá}

Ak systém nereaguje alebo padá, môže to byť spôsobené únikom pamäte alebo konfliktom zdrojov. Skúste reštartovať systém a ak problém pretrváva, skontrolujte, či sa v systémových protokoloch nenachádzajú chybové hlásenia. Na identifikáciu problému môže byť užitočné aj spustenie systému v režime ladenia.

Ak sa počas používania systému vyskytnú iné problémy, pozrite si dokumentáciu alebo kontaktujte náš tím podpory, ktorý vám poskytne ďalšiu pomoc.

\newpage
\begin{flushleft}

\def\refname{Zoznam použitej literatúry}
\addcontentsline{toc}{section}{\numberline{}Zoznam použitej literatúry}

\begin{thebibliography}{999}
    \harvarditem{Meier}{2018}{meier2018dji} Meier, L., 2018. DJI Tello - SDK and Accessory Development. DJI. Available at: \url{https://dl-cdn.ryzerobotics.com/downloads/Tello/Tello\%20SDK\%202.0\%20User\%20Guide.pdf} [Accessed 12-Apr-2023].
    \harvarditem{Socket.IO}{2023}{socketio2023} Socket.IO, 2023. Socket.IO — Documentation. Available at: \url{https://socket.io/docs/v4} [Accessed 12-Apr-2023].
\end{thebibliography}

\end{flushleft}

% \section{Zoznam príloh}

% Prílohy obsahujú výpisy zdrojových textov programov. Každý zdrojový text zacína s~komentárovou hlavickou obsahujúcou nasledujúce informácie:

% Súbor: názov súboru

% Program: názov programu

% Vypracoval:  meno diplomanta

% Diplomová práca: názov diplomovej práce

% Vedúci diplomovej práce: meno vedúceho diplomovej práce

% Konzultant(i): meno konzultanta c.1, meno konzultanta c.2
 
% \lstinputlisting[language=Python, caption=commander.py]{code/commander.py}  
% \lstinputlisting[language=Python, caption=robot\char`_enable\char`_follower.py]{code/robot_enable_follower.py}
% \lstinputlisting[language=Python, caption=robot\char`_enable\char`_lead.py]{code/robot_enable_lead.py}
% \lstinputlisting[language=Python, caption=robot\char`_enable.py]{code/robot_enable.py}
% \lstinputlisting[language=Python, caption=robot\char`_enable\char`_lead.py]{code/teleop_keyboard_listen.py}
% \lstinputlisting[language=Python, caption=teleop\char`_keyboard.py]{code/teleop_keyboard.py}


% \addcontentsline{toc}{section}{\numberline{}Zoznam obrázkov}
% \listoffigures

% \addcontentsline{toc}{section}{\numberline{}Zoznam tabuliek}
% \listoftables
 


\end{document}
